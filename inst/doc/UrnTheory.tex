\documentclass[a4paper]{article}

%\VignetteIndexEntry{Biased Urn Theory}
%\VignettePackage{BiasedUrn}

\usepackage{amsmath}
\usepackage{amssymb}
%
\usepackage{c:/R/share/texmf/Sweave}
\begin{document}

\title{Biased Urn Theory}
\author{Agner Fog}

\maketitle

\section{Introduction}
%
Two different probability distributions are both known in the
literature as ``the'' noncentral hypergeometric distribution. These
two distributions will be called Fisher's and Wallenius' noncentral
hypergeometric distribution, respectively. 

Both distributions can be associated with the classical experiment 
of taking colored balls at random from an urn without replacement. 
If the experiment is unbiased then the result will follow the well-known
hypergeometric distribution. If the balls have different size or weight
or whatever so that balls of one color have a higher probability of being
taken than balls of another color then the result will be a
noncentral hypergeometric distribution.

The distribution depends on how the balls are taken from the urn. 
Wallenius' noncentral hypergeometric distribution is obtained if $n$ 
balls are taken one by one. Fisher's noncentral hypergeometric 
distribution is obtained if balls are taken independently of each other.

Wallenius' distribution is used in models of natural selection and biased
sampling. Fisher's distribution is used mainly for statistical tests in
contingency tables. Both distributions are supported in the {\tt BiasedUrn}
package.

The difference between the two noncentral hypergeometric distributions 
is difficult to understand. I am therefore providing a detailed
explanation in the following sections.


\section{Definition of Wallenius' noncentral hypergeometric distribution}
%
Assume that an urn contains $N$ balls of $c$ different colors and let
$m_i$ be the number of balls of color $i$. Balls of color $i$ have the
weight $\omega_i$. $n$ balls are drawn from the urn, one by one, in 
such a way that the probability of taking a particular ball at a 
particular draw is equal to this ball's fraction of the total weight of
all balls that lie in the urn at this moment.

The color of the $n$ balls that are taken in this way will follow Wallenius'
noncentral hypergeometric distribution. This distribution has the 
probability mass function:
%
$$
\operatorname{dMWNCHypergeo}(\boldsymbol{x};\boldsymbol{m},n,\boldsymbol{\omega}) 
\:=\: 
\left( \prod_{i=1}^c \binom{m_i}{x_i} \right)
\: \int_0^1 \prod_{i=1}^c
(1-t^{{\omega_i}/{d}})^{x_i} \, \mathrm{d}t \;,
$$
%
$$
\text{where } \: d \:=\:
\sum_{i=1}^c \omega_i(m_i-x_i) \,.
$$
%
$\boldsymbol{x}=(x_1,x_2,\ldots,x_c)$ 
is the number of balls drawn of each color.\\
$\boldsymbol{m}=(m_1,m_2,\ldots,m_c)$ 
is the initial number of balls of each color in the urn.\\
$\boldsymbol{\omega}=(\omega_1,\omega_2,\ldots,\omega_c)$
is the weight or odds of balls of each color.\\
$n = \sum_{i=1}^c x_i$ is the number of balls drawn.\\
$c$ is the number of colors. The unexpected integral in this 
formula arises as the solution to a difference equation.
(The above formula is invalid in the trivial case $n = N$.)


\section{Definition of Fisher's noncentral hypergeometric distribution}
%
If the colored balls are taken from the urn in such a way that 
the probability of taking a particular ball of color $i$ is 
proportional to its weight $\omega_i$ and the probability for
each particular ball is independent of what happens to the 
other balls, then the number of balls taken will follow a
binomial distribution for each color.

The total number of balls taken $n = \sum_{i=1}^c x_i$ is
necessarily random and unknown prior to the experiment.
After the experiment, we can determine $n$ and calculate the
distribution of colors for the given value of $n$.
This is Fisher's noncentral hypergeometric distribution, which
is defined as the distribution of independent binomial variates
conditional upon their sum $n$.

The probability mass function of Fisher's noncentral hypergeometric 
distribution is given by
%
$$
\operatorname{dMFNCHypergeo}(\boldsymbol{x};\boldsymbol{m},n,\boldsymbol{\omega})
\:=\:
\frac{\textrm{g}(\boldsymbol{x};\boldsymbol{m},n,\boldsymbol{\omega})}
{\sum\limits_{\boldsymbol{y}\in \: \Xi}
\textrm{g}(\boldsymbol{y};\boldsymbol{m},n,\boldsymbol{\omega})}\:,
$$
%
$$
\text{where } \: \textrm{g}(\boldsymbol{x};\boldsymbol{m},n,\boldsymbol{\omega})
\:=\: \prod_{i=1}^c
\binom{m_i}{x_i}\omega_i^{\,x_i}\:,
$$
%
$$
\text{and the domain }\: \Xi \:=\: \left\{\boldsymbol{x}\in\mathbb{Z}^c \,\middle|\,
\sum_{i=1}^c x_i = n \: \wedge \: 
\forall\, i \in [1,c] \: : \: 0 \leq x_i \leq m_i \right\}\:.
$$


\section{Univariate distributions}
%
The univariate distributions are used when the number of colors
$c$ is $2$. The multivariate distributions are used when the number 
of colors is more than $2$.

The above formulas apply to any number of colors $c$. The univariate
distributions can be expressed by setting 
$c=2$, $\:x_1=x$, $\:x_2=n-x$, $\:m_1=m$, $\:m_2=N-m$,
$\:\omega_1=\omega$, $\:\omega_2=1$.


\section{Name confusion}
Wallenius' and Fisher's distribution are both known in the literature
as ``the'' noncentral hypergeometric distribution. Fisher's distribution
was first given the name extended hypergeometric distribution, but
some scientists are strongly opposed to using this name.

There is a widespread confusion in the literature because these two
distributions have been given the same name and because it is not obvious that
they are different. Several publications have used the wrong distribution
or erroneously assumed that the two distributions were identical.

I am therefore recommending to use the prefixes Wallenius' and Fisher's
to distinguish the two noncentral hypergeometric distributions. While this
makes the names rather long, it has the advantage of emphasizing that there
is more than one noncentral hypergeometric distribution, whereby the
risk of confusion is minimized. 
Wallenius and Fisher are the names of the scientists who first described each of
these two distributions.

The following section explains why the two distributions are different and
how to decide which distribution to use in a specific situation.


\section{The difference between the two distributions}
%
Both distributions degenerate into the well-known hypergeometric distribution
when all balls have the same weight. In other words: It doesn't matter how
the balls are sampled if the balls are unbiased. Only if the urn experiment
is biased can we get different distributions depending on how the balls are
sampled.

It is important to understand how this dependence on the sampling procedure
arises. In the Wallenius model, there is competition between the balls.
The probability that a particular ball is taken is lower when the other 
balls in the urn are heavier. The probability of taking a particular ball
at a particular draw is equal to its fraction of the total weight of the
balls that remain in the urn at that moment. This total weight 
depends on the weight of
the balls that have been removed in previous draws. Therefore, each draw
except the first one has a probability distribution that depends on the
results of the previous draws. The fact that each draw depends on the 
previous draws is what makes Wallenius' distribution unique and makes the
calculation of it complicated. What happens to each ball depends on what
has happened to other balls in the preceding draws.

In the Fisher model, there is no such dependence between draws. We may 
as well take all $n$ balls at the same time. Each ball
has no ``knowledge'' of what happens to the other balls. For the same 
reason, it is impossible to know the value of $n$ before the experiment.
If we tried to fix the value of $n$ then we would have no way of 
preventing ball number $n+1$ from being taken without violating the principle
of independence between balls. $n$ is therefore a random variable and
the Fisher distribution is a conditional distribution which can only
be determined after the experiment when $n$ is known. The unconditional
distribution is $c$ independent binomials.

The difference between the two distributions is low when odds ratios are
near 1, and $n$ is low compared to $N$. The difference between the two 
distributions becomes higher when odds ratios are high and $n$ is near $N$.

Consider the extreme example where an urn contains one red ball with the
weight 1000, and a thousand white balls each with the weight 1. 
We want to calculate the probability that the red ball is not taken.
The probability that the red ball is not taken in the first draw is 
$\frac{1000}{2000} = \frac 12$. The probability that the red ball is
not taken in the second draw, under the condition that it was not taken
in the first draw, is $\frac{999}{1999} \approx \frac 12$.
The probability that the red ball is
not taken in the third draw, under the condition that it was not taken
in the first two draws, is $\frac{998}{1998} \approx \frac 12$. 
Continuing in this way, we can calculate that the probability of not
taking the red ball in $n$ draws is approximately $2^{-n}$. 
In other words, the probability of not taking a very heavy ball in $n$ 
draws falls almost exponentially with $n$ in Wallenius' model.
The exponential function arises because the probabilities for each draw
are all multiplied together.

This is not the case in Fisher's model where balls may be taken 
simultaneously. Here the draws are independent
and the probabilities are therefore not multiplied together. The 
probability of not taking the heavy red ball in Fisher's model is approximately
$\frac{1}{n+1}$. The two distributions are therefore very different
in this extreme case.
\vskip 5mm

The following conditions must be fulfilled for Wallenius' distribution
to be applicable:
%
\begin{itemize}
%
\item Items are taken randomly from a finite source containing different
kinds of items without replacement.
%
\item Items are drawn one by one.
%
\item The probability of taking a particular item at a particular draw is equal
to its fraction of the total weight of all items that have not yet been taken at that
moment. The weight of an item depends only on its kind (color) $i$.
(It is convenient to use the word ``weight'' for $\omega_i$ even if the
physical property that determines the odds is something else than weight).
%
\item The total number $n$ of items to take is fixed and independent of
the weights of the items that are taken.
%
\end{itemize}
\vskip 5mm

The following conditions must be fulfilled for Fisher's distribution
to be applicable:
%
\begin{itemize}
%
\item Items are taken randomly from a finite source containing different
kinds of items without replacement.
%
\item Items are taken independently of each other. Whether one item is taken
is independent of whether another item is taken. Whether one item is taken
before, after, or simultaneously with another item is irrelevant.
%
\item The probability of taking a particular item is proportional to its weight.
The weight of an item depends only on its kind (color) $i$.
%
\item The total number $n$ of items that will be taken is not known 
before the experiment.
%
\item $n$ is determined after the experiment and the conditional distribution
for $n$ known is desired.
%
\end{itemize}


\section{Examples}
%
The following examples will further clarify which distribution to use in different
situations.

\subsection{Example 1}
You are catching fish in a small lake that contains a limited number of fish. 
There are different kinds of fish with different weights. The probability of
catching a particular fish is proportional to its weight when you only catch
one fish.

You are catching the fish one by one with a fishing rod. You have been ordered
to catch $n$ fish. You are determined to catch exactly $n$ fish regardless of
how long time it may take. You are stopping after you have caught $n$ fish
even if you can see more fish that are tempting you.

This scenario will give a distribution of the types of fish caught that is equal to 
Wallenius' noncentral hypergeometric distribution.

\subsection{Example 2}
You are catching fish as in example 1, but you are using a big net.
You are setting up the net one day and coming back the next day to
remove the net. You count how many fish you have caught and then you go 
home regardless of how many fish you have caught.

Each fish has a probability of getting into the net that is proportional
to its weight but independent of what happens to the other fish.

This scenario gives Fisher's noncentral hypergeometric distribution after
$n$ is known.

\subsection{Example 3}
You are catching fish with a small net. It is possible that more than one
fish can go into the net at the same time. You are using the net multiple
times until you have at least $n$ fish.

This scenario gives a distribution that lies between Wallenius' and Fisher's
distributions. The total number of fish caught can vary if you are getting too 
many fish in the last catch. You may put the excess fish back into the lake,
but this still doesn't give Wallenius' distribution. This is because you
are catching multiple fish at the same time. The condition that each catch
depends on all previous catches does not hold for fish that are caught 
simultaneously or in the same operation.

The resulting distribution will be close to Wallenius' distribution if 
there are only few fish in the net in each catch and you are catching
many times.

The resulting distribution will be close to Fisher's distribution if 
there are many fish in the net in each catch and you are catching
few times.

\subsection{Example 4}
You are catching fish with a big net. Fish are swimming into the net
randomly in a situation that resembles a Poisson process. You are
watching the net all the time and take up the net as soon as you have
caught exactly $n$ fish.

The resulting distribution will be close to Fisher's distribution
because the fish swim into the net independently of each other.
But the fates of the fish are not totally independent because 
a particular fish can be saved from getting caught if $n$ other 
fish happen to get into the net before the time that this particular
fish would have been caught. This is more likely to happen if the other
fish are heavy than if they are light.

\subsection{Example 5}
You are catching fish one by one with a fishing rod as in example 1.
You need a particular amount of fish in order to feed your family.
You are stopping when the total weight of the fish you have caught
exceeds a predetermined limit.

The resulting distribution will be close to Wallenius' distribution,
but not exact because the decision to stop depends on the weight of
the fish you have caught so far. $n$ is therefore not known exactly
before the fishing trip.

\subsection{Conclusion}
These examples show that the distribution of the types of 
fish you catch depends on the way they are caught. Many situations
will give a distribution that lies somewhere between Wallenius'
and Fisher's noncentral hypergeometric distributions.

An interesting consequence of the difference between these two
distributions is that you will get more of the heavy fish, on average, 
if you catch $n$ fish one by one than if you catch all $n$
at the same time.

These conclusions can of course be applied to biased sampling of
other items than fish.

\section{Demos}
%
The following demos are included in the {\tt BiasedUrn} package:

\subsection{\tt CompareHypergeo}
%
This demo shows the difference between the hypergeometric distribution
and the two noncentral hypergeometric distributions by plotting
the probability mass functions.

\subsection{\tt ApproxHypergeo}
%
This demo shows shows that the two noncentral hypergeometric 
distributions are approximately equal when the parameters are
adjusted so that they have the same mean rather than the same odds.

\subsection{\tt OddsPrecision}
%
Calculates the precision of the {\tt oddsWNCHypergeo} and {\tt oddsFNCHypergeo}
functions that are used for estimating the odds from a measured mean.

\subsection{\tt SampleWallenius}
%
Makes 100,000 random samples from Wallenius noncentral hypergeometric 
distribution and compares the measured mean with the theoretical mean.

\subsection{\tt UrnTheory}
%
Displays this document.


\section{Calculation methods}
%
The {\tt BiasedUrn} package can calculate the univariate 
and multivariate
Wallenius' and Fisher's noncentral hypergeometric distributions.
Several different calculation methods are used, depending on the 
parameters.

The calculation methods and sampling methods are documented at \\
{\tt http://www.agner.org/random/theory/}. An article with a formal
description has been submitted to 
\begin{emph}{Communications in Statistics: Simulation and Computation}\end{emph}.

\section{References}

\noindent Johnson, N. L., Kemp, A. W. Kotz, S. (2005). {\it
Univariate Discrete Distributions}. Hoboken, New Jersey: Wiley and
Sons.

\vskip 3mm
%
\noindent McCullagh, P., Nelder, J. A. (1983). {\it Generalized
Linear Models}. London: Chapman \& Hall.

\vskip 3mm
%
\noindent {\tt http://www.agner.org/random/theory/}.


\end{document}
